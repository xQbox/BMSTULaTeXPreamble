\documentclass[a4paper, 12pt]{report}

% Поддержка языков
\usepackage[english, russian]{babel} 

% Настройка кодировок
\usepackage[T2A]{fontenc}
\usepackage[utf8]{inputenc}

% Настройка шрифтов
\usepackage{fontspec}
\setmainfont[Ligatures=TeX]{Times New Roman} % Шрифт для основного текста документа
\setsansfont[Ligatures=TeX]{Arial}
\setmonofont{Consolas} % Шрифт для кода

% Настройка отступов от краев страницы
\usepackage[left=3cm, right=1.5cm, top=2cm, bottom=2cm]{geometry}

\usepackage{titleps} % Колонтитулы
\usepackage{subfig} % Для подписей к рисункам и таблицам
\usepackage{graphicx} % для вставки картинок
\graphicspath{{./img/}} % Путь до папки с изображениями
\usepackage[backend=bibtex, style=numeric]{biblatex} % Использование BibTeX через biblatex
\addbibresource{./inc/books.bib} % Путь до файла .bib с книгами

% Пакет для отрисовки графиков
\usepackage{tikz}
\usetikzlibrary{arrows,positioning,shadows}
\usepackage{stmaryrd} % Стрелки в формулах
\usepackage{indentfirst} % Красная строка после заголовка
\usepackage{hhline} % Улучшенные горизонтальные линии в таблицах
\usepackage{multirow} % Ячейки в несколько строчек в таблицах
\usepackage{longtable} % Многостраничные таблицы
\usepackage{paralist,array} % Список внутри таблицы
\usepackage{threeparttable} % Таблицы с примечаниями, подписями и ссылками
\usepackage[normalem]{ulem}  % Зачеркнутый текст
\usepackage{upgreek, tipa} % Красивые греческие буквы
\usepackage{amsmath, amsfonts, amssymb, amsthm, mathtools} % ams пакеты для математики, табуляции
\usepackage{nicematrix} % Особые матрицы pNiceArray


\linespread{1.5} % Межстрочный интервал
\setlength{\parindent}{1.25cm} % Табуляция
\setlength{\parskip}{0cm}

% Пакет для красивого выделения кода
\usepackage{minted}
\setminted{fontsize=\footnotesize}

% Добавляем гипертекстовое оглавление в PDF
\usepackage[
bookmarks=true, colorlinks=true, unicode=true,
urlcolor=black,linkcolor=black, anchorcolor=black,
citecolor=black, menucolor=black, filecolor=black,
]{hyperref}

% Убрать переносы слов
\tolerance=1
\emergencystretch=\maxdimen
\hyphenpenalty=10000
\hbadness=10000

\newpagestyle{main}{
	% Верхний колонтитул
	\setheadrule{0cm} % Размер линии отделяющей колонтитул от страницы
	\sethead{}{}{} % Содержание {слева}{по центру}{справа}
	% Нижний колонтитул
	\setfootrule{0cm} % Размер линии отделяющей колонтитул от страницы
	\setfoot{}{}{\thepage} % Содержание {слева}{по центру}{справа}
}
\pagestyle{main}

% НОВЫЕ КОМАНДЫ
\newcommand{\deriv}[2]{\frac{\partial #1}{\partial #2}}
\newcommand{\n}{\par}
\newcommand{\percent}{\mathbin{\%}}

% Заменяем Рис. на Рисунок
\addto\captionsrussian{\renewcommand{\figurename}{Рисунок}}

% Изменение формата подписей
% Стиль номера таблицы/рисунка #1-Таблица/Рис. #2-номер
\DeclareCaptionLabelFormat{custom}
{%
	#1 #2
}
% Стиль разделителя номера таблици/рисунка и названия таблицы/рисунка
\DeclareCaptionLabelSeparator{custom}{$-$}
% Стиль формата #1-номер таблицы/рисунка #2-разделитель #3-название
\DeclareCaptionFormat{custom}
{%
	#1 #2 #3
}

\captionsetup
{
	format=custom,%
	labelsep=custom,
	labelformat=custom
}

% ПЕРЕГРУЗКА УЖЕ СУЩЕСТВУЮЩИХ КОМАНД
\renewcommand{\epsilon}{\varepsilon} % Заменить знак эпсилон
\renewcommand{\phi}{\varphi}
\renewcommand{\kappa}{\varkappa}
\renewcommand{\lambda}{\uplambda}