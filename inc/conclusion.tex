\chapter*{Заключение}
\addcontentsline{toc}{chapter}{Заключение}

Экспериментально было подтверждено различие во временной эффективности рекурсивной и нерекурсивной реализаций алгоритмов вычисления расстояния между строками с помощью разработанного программного обеспечения на основе измерений процессорного времени для строк разной длины.

Исследования показали, что матричная реализация этих алгоритмов значительно превосходит рекурсивную по времени выполнения при увеличении длины строк, но требует больше памяти.

\vspace{5mm}

В ходе выполнения данной лабораторной работы были решены следующие задачи:
В ходе выполнения данной лабораторной работы были решены следующие задачи:
\begin{itemize}
	\item изучены алгоритмов Левенштейна и Дамерау-Левенштейна нахождения расстояния между строками;
	\item применены методы динамического программирования для матричной реализации указанных алгоритмов;
	\item получены практические навыки реализации указанных алгоритмов: двух алгоритмов в матричной версии и одного из алгоритмов в рекурсивной версии;
	\item проведен сравнительный анализ линейной и рекурсивной реализаций выбранного алгоритма определения расстояния между строками по затрачиваемым ресурсам (времени и памяти);
	\item описаны и обоснованы полученные результаты в отчете о выполненной лабораторной работе, выполненного как расчётно-пояснительная записка к работе.
\end{itemize}